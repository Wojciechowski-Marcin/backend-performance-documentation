\documentclass[mgr, oneside, en]{mgr}
\usepackage{listingsutf8}

\usepackage{booktabs, multirow} % for borders and merged ranges
\usepackage{soul}% for underlines
\usepackage[table]{xcolor} % for cell colors
\usepackage{changepage,threeparttable} % for wide tables

\usepackage[english]{babel}
\usepackage[utf8]{inputenc}
\usepackage[T1]{fontenc}
\usepackage[ruled,vlined]{algorithm2e}

\usepackage{graphicx}
\usepackage{subfigure}
\usepackage{psfrag}

\usepackage{amsmath}
\usepackage{amsfonts}

\usepackage{supertabular}
\usepackage{array}
\usepackage{tabularx}
\usepackage{hhline}

\usepackage{listings}
\usepackage{rotating}
\usepackage{graphicx}
\usepackage{setspace}
\usepackage{caption}
\usepackage{float}

\usepackage[utf8]{inputenc}
\usepackage[acronym]{glossaries}

\makenoidxglossaries

% TODO use all acronyms

\newacronym{pg}{PG}{PostgreSQL}
\newacronym{vu}{VU}{Virtual User}
\newacronym{orm}{ORM}{Object-Relational Mapping}
\newacronym{mvc}{MVC}{Model-View-Controller}
\newacronym{wsgi}{WSGI}{Web Server Gateway Interface}
\newacronym{asgi}{ASGI}{Asynchronous Server Gateway Interface}
\newacronym{npm}{NPM}{Node Package Manager}
\newacronym{js}{JS}{JavaScript}
\newacronym{crud}{CRUD}{Create Retrieve Update Delete}
\newacronym{http}{HTTP}{Hypertext Transfer Protocol}
\newacronym{api}{API}{Application Programming Interface}
\newacronym{drf}{DRF}{Django Rest Framework}
\newacronym{sql}{SQL}{Structured Query Language}
\newacronym{json}{JSON}{JavaScript Object Notation}
\newacronym{lts}{LTS}{Long Term Support}
\newacronym{db}{DB}{Database}
\newacronym{csv}{CSV}{Comma-Separated Values}
\newacronym{cors}{CORS}{Cross-Origin Resource Sharing}
\newacronym{owasp}{OWASP}{Open Web Application Security Project}


\renewcommand{\labelenumii}{\theenumii}
\renewcommand{\theenumii}{\theenumi.\arabic{enumii}.}

\newfloat{pseudocode}{thp}{lop}[chapter]
\floatname{pseudocode}{Pseudokod}

\input{dockerlistings.tex}

%pakiet wypisujący na marginesie etykiety równań i rysunków zdefiniowanych przez \label{}, chcąc wygenerować finalną wersję dokumentu wystarczy usunąć poniższą linię
% \usepackage{showlabels}

%definicje własnych poleceń
\newcommand{\R}{I\!\!R} %symbol liczb rzeczywistych, działa tylko w trybie matematycznym
\newtheorem{theorem}{Twierdzenie}[section] %nowe otoczenie do składania twierdzeń
\renewcommand\lstlistlistingname{List of Code Listings}

% STRONA TYTULOWA
\title{Analiza porównawcza i ocena wydajności frameworków back-endowych w
aplikacjach bazodanowych}
\engtitle{Comparison analysis and efficiency evaluation of back-end frameworks in
database applications}
\author{Marcin Wojciechowski}
\supervisor{Dr inż. Paweł Głuchowski W4/K9}
%\guardian{Dr inż. Paweł Trajdos} %nie używać jeśli opiekun jest tą samą osobą co prowadzący pracę

\date{2021}

\field{Informatyka Techniczna (INF)} % TODO find INF english name

\specialisation{Internet Engineering (INE)}

\begin{document}

\bibliographystyle{ieeetr}

\maketitle

% DEDYKACJA
% \dedication{6cm}{To jest przykładowa treść opcjonalnej dedykacji, należy ją zmienić lub usunąć w całości polecenie \texttt{$\backslash$dedication}}

\tableofcontents

%
%
% TRESC PRACY
%
%


\addcontentsline{toc}{chapter}{Acronyms}
\printnoidxglossaries

% !TeX root = ./0Base.tex

\chapter{Introduction}

Complex web applications keep becoming more popular in recent years. They are very easily accessible from any place in the world and can run on almost any modern device, as it requires only web browser and the Internet connection to run.

Web frameworks simplify the development process of web applications significantly improving developers productivity. There is a huge variety of choices between the mentioned software, so choosing one may be a difficult task.

The choice of server-side framework is crucial, as it is responsible for handling sensitive data and plays a key role in the overall performance of the application.

The goal of this document is to focus on three of the most popular server-side frameworks and compare their performance under high load as well as basic security measurements. The results of this thesis will be important for web developers and architects, that need to decide on which framework should they choose for their application.

Web frameworks for this comparison were chosen from the list of Stack Overflow Developer Survey 2020 Web Frameworks popularity \cite{devSurveyWeb}. The chosen frameworks (and their respective languages) are:
\begin{itemize}
    \item Express.js (JavaScript)
    \item ASP.NET (C\#)
    \item Django (Python)
\end{itemize}

Chapter 2 describes chosen frameworks, including their architecture and requirements. For the definition of the criteria against which the results will be measured, see Chapter 3. Chapter 4 presents design of the system - database models, application structure and environment preparation. Chapter 5 shows framework specific application implementation details. Results of the tests and comparison of the technologies can be found in chapter 6. For final conclusion of this document, see chapter 7.


% !TeX root = ./0Base.tex

\chapter{Technology description}

\section{Django}

\subsection{Overview}
Django lets you build deep, dynamic, interesting sites in an extremely short time. Django is designed to let you focus on the fun, interesting parts of your job while easing the pain of the repetitive bits \cite{djangobook}. Additionally the framework is being supported by a wide variety of libraries and frameworks, like Django Rest Framework, Django Celery, Crispy Forms.

\subsection{Architecture}
Django is based on MVC architecture:
\begin{itemize}
    \item Model - a data structure, represented by a database
    \item View - responses visible in the browser
    \item Controller - connects Model and View together - describes how the data should be presented to the user
\end{itemize}
In Django application there are multiple files and at first it may not be obvious what their role is. Basic structure looks like this:
\begin{itemize}
    \item apps.py - common to all django apps configuration file
    \item models.py - custom models
    \item serializers.py - define how our model objects should be converted into response
    \item views.py - custom controllers, which is unintuitive for most of people; as the developers explain, in their interpretation of MVC the view describes which data gets presented to the user \cite{djangoWhyViews}
    \item urls.py - defines which endpoint responds to given controller
\end{itemize}
Templates, which are not mentioned above, are the Django's custom views - in our case, I am going to be using build in json parsers.

\subsection{Requirements}
To install and run a simple Django project, two main things are required:
\begin{itemize}
    \item pip (easiest install method)
    \item Python
\end{itemize}


\section{ExpressJS}

\subsection{Overview}
\subsection{Architecture}
\subsection{Requirements}


\section{ASP.NET}

\subsection{Overview}
\subsection{Architecture}
\subsection{Requirements}

\section{K6 and related tools}

For testing the performance of applications, I chose a tool named k6. It is a modern load testing tool written in Golang, which provides clean and well documented APIs for writing and running tests, while still being easily configurable to the developers needs. Test logic and configuration options are both to be written in JavaScript, which allows developers for using JavaScript modules, which aids in code reusability. The creators of k6 prepared two types of execution:
\begin{itemize}
    \item local, through command line interface
    \item and cloud, which is a commercial SaaS product, made to make performance testing in bigger applications easier.
\end{itemize}
For the sake of this experiment, local testing has fulfilled all expectations.

Installation on Ubuntu operating system is fairly simple and all necessary commands were described in the documentation. However, to make the testing simpler, a k6 Docker image was used, that together with Docker Compose allowed to create a single script that would handle all test cases as described in the following section.

K6 allows to create visualizations, using built-in InfluxDB and Grafana integration, where InfluxDB is used as storage backend and Grafana to visualize the data. In this research, only InfluxDB was added to store the data and after each test the data was exported to file, which later allowed to compare the results between applications on a single chart.

\section{Docker and Docker Compose}

Docker and Docker Compose were used to simplify the development. This made starting all services that had be run together possible with only one command, eg. for django performance tests - django, postgres, k6 and influx. For every application a production ready Dockerfile was created. Additionally, Docker provides applications a layer of isolation from each other and the host.

\section{PostgreSQL}

For the Database Management System I chose PostgreSQL, which is the second most popular choice among database technologies %TODO link to dev survey


% !TeX root = ./0Base.tex

\chapter{Criteria description}

\section{Performance benchmark}

\subsection{Scenarios}
The task for each application is to complete simple CRUD operations as fast as possible. For comparison of the performance of the applications, a few scenarios were developed:

\begin{itemize}
    \item retrieving multiple objects (getMany)
    \item retrieving single object (get)
    \item updating a single object (put)
    \item creating a single object (post)
    \item deleting a single object (delete)
\end{itemize}

Scenarios were tested with a few different application loads, which are represented by a number of virtual users (VUs) - as mentioned in the k6 repository description they are glorified, parallel while(true) loops.
% TODO annotation to k6 repo
The numbers chosen for tests are:
\begin{itemize}
    \item 1 VU
    \item 8 VUs
    \item 32 VUs
    \item 128 VUs
    \item 512 VUs
\end{itemize}
For a single virtual user case the number of concurrences does not change throughout the duration of the test, however, as suggested in
% TODO find article that suggests this
, for bigger numbers of concurrent users, the tests should include warmup and cooldown period. All tests are 45 seconds long, and tests with more than 1 virtual user include 15 seconds of ramp up time and 15 seconds of ramp down time as shown on figure \ref{fig:vusPerSecond}.

Longer test times did not bring any valuable information and only brought CPU overheating problems, thus they were shortened, which also made work with the results much easier.

\begin{figure}[H]
    \includegraphics[width=\columnwidth]{pictures/vusPerSecond.png}
    \caption{Amount of VUs per second during the tests}
    \label{fig:vusPerSecond}
\end{figure}


\subsection{Database snapshots}

To avoid any differences in the database between the tests, at the beginning of the tests the database is populated and the snapshot is stored locally. Before each test, the snapshot is restored.

\subsection{Application isolation}

To be sure that the applications are running in an isolated environment, docker containers were used. The configuration prepared for the applications included environment preparation (installing necessary packages, providing environment variables), To simplify the research, a Docker Compose configuration was prepared, that builds and starts all the necessary containers at once.

\subsection{Test progress}

The measures are gathered from each application in the following manner:

\begin{algorithm}[H]
    \label{alg:testProgress}
    \caption{Pseudocode describing load testing process}

    frameworks = [aspnet django express]\;
    scenarios = [1 8 32 128 512]\;
    iterations = [1 2 3 4 5 6 7 8 9 10]\;
    test cases = [get post put delete getMany]\;

    \ForAll{frameworks}{
        start application and database\;
        populate database\;
        kill application and database\;
        store snapshot\;
        \ForAll{scenarios}{
            \ForAll{iterations}{
                \ForAll{test cases}{
                    \If{application is running}{kill application and database\;}
                    remove volumes\;
                    \If{test case is not post}{restore snapshot\;}
                    start application and database\;
                    \While{application is not responding}{wait for application\;}
                    \For{30 seconds}{measured test\;}
                    store k6 result \;
                    store influxDB result \;
                    \For{30 seconds}{cooldown before next test\;}
                }
            }
        }
    }
    merge results\;
\end{algorithm}


\subsection{Software versions and hardware}

The tests were run on a laptop with the specification presented in table \ref{tab:hardware}.
Frameworks used to build the application were in the versions presented in table \ref{tab:software}.

\begin{table}[!htp]\centering
    \caption{Hardware}\label{tab:hardware}
    \scriptsize
    \begin{tabular}{lrr}\toprule
        Hardware         &                                          \\\midrule
        Processor        & Intel(R) Core(TM) i5-8250U CPU @ 1.60GHz \\
        RAM memory       & 16 GB @ 2400MT/s                         \\
        Operating system & Ubuntu 20.04.2 LTS                       \\
        \bottomrule
    \end{tabular}
\end{table}



\FloatBarrier
\begin{table}[!htp]\centering
    \caption{Frameworks and libraries versions}\label{tab:software}
    \scriptsize
    \begin{tabular}{lrr}\toprule
        Software versions                     &         \\\midrule
        Python                                & 3.1.9   \\
        Django                                & 3.1.4   \\
        Django REST Framework                 & 3.12.2  \\
        gunicorn                              & 20.0.4  \\
        uvicorn                               & 0.13.1  \\\midrule
        C\#                                   & 7.3     \\
        ASP.NET                               & 2.1.1   \\
        Npgsql.EntityFrameworkCore.PostgreSQL & 2.1.1.1 \\\midrule
        Node.js                               & 15.12.0 \\
        Express                               & 4.17.1  \\
        pg-promise                            & 10.9.5  \\
        body-parser                           & 1.19.0  \\
        \bottomrule
    \end{tabular}
\end{table}
\FloatBarrier


\section{Security}


% !TeX root = ./0Base.tex

\chapter{System design}

\section{Database}

For the tests, a database consisting of a single table was created. 

\section{Application}
\section{Environment}


% !TeX root = ./0Base.tex

\chapter{Application implementation}

\section{Django}
\section{Express}
\section{ASP.NET}


% !TeX root = ./0Base.tex

\chapter{Results}\label{cha:results}

\section{Performance}

Table \ref{tab:resultsFromFile} shows an average response time from the 10 tests mentioned in the previous chapters.

\begin{table}[!htp]\centering
    \caption{Average p(95) response time in tests}\label{tab:resultsFromFile}
    \scriptsize
    \begin{tabular}{lrrrrr}\toprule
        AVERAGE z p(95) &             & filename &          &         \\\midrule
        test            & concurrency & aspnet   & django   & express \\
        delete          & 1           & 8.35     & 48.36    & 4.14    \\
                        & 8           & 9.75     & 143.84   & 10.94   \\
                        & 32          & 48.84    & 385.77   & 32.68   \\
                        & 128         & 192.80   & 1259.32  & 118.67  \\
                        & 512         & 675.12   & 4923.61  & 458.99  \\
        get             & 1           & 1.37     & 12.54    & 1.00    \\
                        & 8           & 6.42     & 54.46    & 5.98    \\
                        & 32          & 19.89    & 189.95   & 20.05   \\
                        & 128         & 84.64    & 643.71   & 79.42   \\
                        & 512         & 309.29   & 2605.04  & 321.64  \\
        getMany         & 1           & 48.16    & 56.93    & 24.14   \\
                        & 8           & 76.09    & 344.77   & 42.59   \\
                        & 32          & 177.36   & 1280.14  & 73.80   \\
                        & 128         & 325.66   & 4831.64  & 159.44  \\
                        & 512         & 683.23   & 19525.67 & 511.82  \\
        patch           & 1           & 9.13     & 45.58    & 5.20    \\
                        & 8           & 10.53    & 129.42   & 11.65   \\
                        & 32          & 50.86    & 358.92   & 35.65   \\
                        & 128         & 198.23   & 1237.40  & 130.69  \\
                        & 512         & 655.22   & 4806.19  & 501.05  \\
        post            & 1           & 6.73     & 40.73    & 5.13    \\
                        & 8           & 7.32     & 122.30   & 10.37   \\
                        & 32          & 42.36    & 321.98   & 29.24   \\
                        & 128         & 171.97   & 1069.47  & 120.47  \\
                        & 512         & 611.54   & 4185.22  & 407.89  \\
        put             & 1           & 9.12     & 45.20    & 5.26    \\
                        & 8           & 10.49    & 132.10   & 11.73   \\
                        & 32          & 51.82    & 359.56   & 34.78   \\
                        & 128         & 202.52   & 1236.24  & 129.99  \\
                        & 512         & 735.95   & 4817.58  & 507.11  \\
        \bottomrule
    \end{tabular}
\end{table}


\section{Security}

\subsection{Security Misconfiguration}
% DJANGO
% https://docs.djangoproject.com/en/3.2/howto/deployment/checklist/
% python3 manage.py check --deploy
% https://hdivsecurity.com/owasp-security-misconfiguration
\subsection{Injection}
% DJANGO
% https://hdivsecurity.com/sql-injection-prevention
% https://docs.djangoproject.com/en/3.2/topics/security/#sql-injection-protection
% Injections can be done using inserting raw() queries
\subsection{Insufficient Logging}
% DJANGO
% https://docs.djangoproject.com/en/3.0/topics/logging/
% https://pypi.org/project/django-automated-logging/
% https://pypi.org/project/django-log-viewer/
% When your application has insufficient logging and monitoring, attacks and suspicious activity can go unnoticed. By default, Django uses the Python native logging module for system logging. 
% django.security.* logging messages.

% ASPNET
% https://www.infoq.com/presentations/owasp-top-10-vulnerabilities-2017/


% !TeX root = ./0Base.tex

\chapter{Conclusion}

% https://neoteric.eu/blog/node-js-vs-python/

% \appendix
% \chapter{Donec cursus nulla vitae pede}

\addcontentsline{toc}{chapter}{\bibname} %utworzenie w spisie treści pozycji Literatura
\bibliography{bibliografia} % wstawia bibliografię korzystając z pliku bibliografia.bib - dotyczy BibTeXa, jeżeli nie korzystamy z BibTeXa należy użyć otoczenia thebibliography
\addcontentsline{toc}{chapter}{\listfigurename}
\listoffigures
\addcontentsline{toc}{chapter}{List of Code Listings}
\lstlistoflistings
\addcontentsline{toc}{chapter}{List of Algorithms}
\listofalgorithms

\end{document}
