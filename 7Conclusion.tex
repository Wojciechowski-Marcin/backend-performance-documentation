% !TeX root = ./0Base.tex

\chapter{Conclusion}

The goal of the thesis was to compare three of the most popular backend frameworks based on performance and security measures. When it comes to performance, there is no definitive winner - it depends on the purpose of the application, as both ASP.NET and Express.js handled the test requests very well. Under heavy load ASP.NET had faster responses than his main opponent, which means that if the goal is to achieve the best performance on a single instance of the application and load is planned to be around 500 or more users requesting the \acrshort{api} at once, it would be the best choice. Express.js would be the best choice if the expected load is going to be smaller than that, or if multiple instances of the service can be started and load can be divided between them through a load balancer. Django is not recommended for high loads, as the responses were very long (especially for retrieving multiple objects from the database, where they got to almost 4 seconds long for the highest load). Speed of Django is highly related to Python, which in most use cases is significantly slower than Node.js, as shown in The Computer Language Benchmarks Game \cite{benchmarksGame}.

When it comes to security, all frameworks passed the injection tests, and all of them provide tools to help the developers keep track of what is going within their application (whether the tools are built in or external). Django has a couple of variables that need to be properly configured before the application is safe for production, however they are clearly marked with a \lstinline{SECURITY WARNING} comment right above them, to make sure after the developer is done with the app creation these variables need to be changed. Django positively surprised with the built in command that checks if the steps to make the environment production-ready were done.

What stood out the most while developing the applications is how many built in functions Django and \acrshort{drf} have, making the process of development very simple. To create the same applications, ASP.NET Core and Express.js required more than twice as much written code than Django. In addition to that, many features of Django were not mentioned or barely mentioned in this document. For example it offers and admin panel to manage the database models in user-friendly \acrshort{gui}, build in User model that is also capable of handling authentication, template engine to create user interfaces, customizable file uploader and storage system, mail sending interface and much more.

To summarize the work and answer the question what each framework is best for:
\begin{itemize}
    \item Django is a very powerful tool for prototyping applications and applications that are expected to have a small users requesting the application at once. It would also be great for developers starting their career in web development, since it provides an extensive documentation and not a lot of code has to be written to make the application work, and on later stages of development almost everything can be customized to fit the needs. Definitely not recommended for real-time applications as the performance is not great, but for small applications it is be more than enough.
    \item Both Express.js and ASP.NET would meet the expectations when creating fast, real-time applications, however for high loads on a single instance ASP.NET would be the better pick.
    \item ASP.NET is better choice than Express.js if external modules are not preferable. It offers high performance while still having a lot of features built in the framework.
    \item Express.js functionalities are limited, but on the other side they can be extended through external packages, which results in every application containing at least a few packages. The creator should pay a special attention to the used packages, as they may contain a security vulnerability that could potentially result in a breach in the system later on. Another thing to remember is that a package can be removed from the repository, and it may not be possible to build the application anymore - to give an example of a tragedy based on this issue, once an open-source developer named Koçulu decided to delete all of his npm packages in a protest and a soon after that developers discovered that their React applications (the most popular frontend framework) do not build anymore - as it turned out, React used one of his 273 packages.
    \item If the stability of the framework is the most important in given use case and the load is expected to be high Express.js should be chosen. If application is mostly based on put requests or the load is small, ASP.NET and Django are to be considered.
    \item For most cases the differences between ASP.NET and Express.js are minor - the choice depends on a specific problem that the application is going to solve.
\end{itemize}

Code of the application was built to easily add more frameworks to the tests - all that needs to be done is creating an application based on the definition from chapter \ref{cha:systemDesign}, adding the docker compose configuration and adding the name of the docker compose profile into the main script variable. More \acrlong{dbms}s can be also added similarly to the applications, but additionally environment would need to be added and used in the applications instead of \acrlong{pg} variables.
